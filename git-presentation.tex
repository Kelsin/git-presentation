\usepackage{color}
\usepackage{float}

\input xy
\xyoption{all}

\usepackage{beamerthemesplit}
\usetheme{Darmstadt}
\usecolortheme{crane}

\setbeamercovered{dynamic}

\title{Git for mere mortals}
\author{Christopher Giroir}
\date{July 26th, 2011}

% Commands
\newcommand{\dia}[1]{\begin{figure}[H]\centerline{\xymatrix{#1}}\end{figure}}

\begin{document}

\begin{frame}
  \titlepage
\end{frame}

\begin{frame}
  \frametitle{Contents}
  \tableofcontents
\end{frame}

\section{Internals}

\subsection{Objects}

\begin{frame}
  \frametitle{Object Store}
  The git object store is a big black box and we like it that way.
\end{frame}

\begin{frame}
  \frametitle{Blobs}
  \dia{ *+++=[o][F-]{\txt{Blob}} }
  \begin{itemize}
  \item Blobs store file data (text or binary).
  \item Labeled by SHA1. \textit{If the content changes, so does the SHA.}
  \end{itemize}
\end{frame}

\begin{frame}
  \frametitle{Trees}
  \dia{ &
    *+++=[o][F-]{\txt{Blob}} \\
    *+++=[o][F-]{\txt{Tree}} \ar[r] \ar[ru] &
    *+++=[o][F-]{\txt{Blob}}
  }
  \begin{itemize}
  \item Trees point to blobs
  \item Represent the entire state of the working directory
  \item Also labeled by SHA's.
  \end{itemize}
\end{frame}

\begin{frame}
  \frametitle{Commits}
  \dia { & &
    *+++=[o][F-]{\txt{Blob}} \\
    *+++=[o][F-]{\txt{Commit}} \ar[r] &
    *+++=[o][F-]{\txt{Tree}} \ar[ru] \ar[r] &
    *+++=[o][F-]{\txt{Blob}}
  }
  \begin{itemize}
  \item Point to a tree object
  \item Stores author and date information
  \item Labeled by SHA's.
  \end{itemize}
\end{frame}

\begin{frame}
  \frametitle{Important Distinction!}
  \begin{itemize}
  \item Commits link to entires \textcolor{blue}{\textbf{TREES}} not
  \textcolor{red}{\textbf{DIFFS}}.
  \item Diffs are not stored in git, they are computed.
  \end{itemize}
\end{frame}

\subsection{Branches and Tags}

\begin{frame}
  \frametitle{Pointers}
  Most other things in git are pointers to commits.
  \dia {
    *++[o][F-]{a} \ar[r] \save +<0cm,0.6cm>*\txt{\textit{v3.0.1}} \restore &
    *++[o][F-]{b} \ar[r] \ar[rd] \save +<0cm,0.6cm>*\txt{Master} \restore &
    *++[o][F-]{c} \ar[r] &
    *++[o][F-]{d} \ar[r] &
    *++[o][F-]{e} \save +<0cm,0.6cm>*\txt{SAVE-465} \restore \\ & &
    *++[o][F-]{g} \ar[r] &
    *++[o][F-]{h} \ar[r] &
    *++[o][F-]{i} \save +<0cm,0.6cm>*\txt{SAVE-576} \restore
  }
  \begin{itemize}
  \item Branches are just labels to commit SHA's
  \item Tags are just labels with meta information
  \end{itemize}
\end{frame}

\begin{frame}
  \frametitle{Rewriting History}
  \begin{itemize}
  \item Rewriting a commit changes it's SHA
    \pause
  \item Any commit after it changes as well
    \pause
  \item Most branch operations are quick and painless
    \pause
  \item Can shoot yourself in the foot and heal it back up again
  \end{itemize}
\end{frame}

\section{Distribution}

\begin{frame}
  \frametitle{Distributed Source Control}
  \begin{itemize}
  \item Only difference from gitorious and local: working directories
    \pause
  \item Can function offline
    \pause
  \item Can ``push'' and ``pull'' from each other, as well as servers
    \pause
  \item Each of our computers serves as a backup of the server
  \end{itemize}
\end{frame}

\begin{frame}
  \frametitle{Location Descriptions}
  Git has 4 locations where commits are stored
  \pause
  \begin{itemize}
  \item \textbf{Remote Repo} Any remote object store
    \pause
  \item \textbf{Local Repo} You local object store
    \pause
  \item \textbf{Index} A single spot to ``stage'' commits
    \pause
  \item \textbf{Working Directory} These are the files you are editing
  \end{itemize}
\end{frame}

\begin{frame}
  \frametitle{Locations and Commands}

  \only<1>{
    \begin{figure}[H]\centerline{\xymatrix{
          *++[F-,]{\txt{Working}} &
          *++[F-,]{\txt{Index}} &
          *++[F-,]{\txt{Local Repo}} &
          *++[F-,]{\txt{Remote Repo}}
  }}\end{figure}}
  \only<2>{
    \begin{figure}[H]\centerline{\xymatrix{
          *++[F-,]{\txt{Working}} \ar@/^2pc/[r]^{\txt{git add}}&
          *++[F-,]{\txt{Index}} \ar@/^2pc/[r]^{\txt{git commit}}&
          *++[F-,]{\txt{Local Repo}} &
          *++[F-,]{\txt{Remote Repo}}
  }}\end{figure}}
  \only<3>{
    \begin{figure}[H]\centerline{\xymatrix{
          *++[F-,]{\txt{Working}} \ar@/^2pc/[r]^{\txt{git add}}&
          *++[F-,]{\txt{Index}} \ar@/^2pc/[r]^{\txt{git commit}}&
          *++[F-,]{\txt{Local Repo}} \ar@/^2pc/[ll]^{\txt{git checkout}}&
          *++[F-,]{\txt{Remote Repo}}
  }}\end{figure}}
  \only<4>{
    \begin{figure}[H]\centerline{\xymatrix{
          *++[F-,]{\txt{Working}} \ar@/^2pc/[r]^{\txt{git add}}&
          *++[F-,]{\txt{Index}} \ar@/^2pc/[r]^{\txt{git commit}}&
          *++[F-,]{\txt{Local Repo}} \ar@/^2pc/[r]^{\txt{git push}} \ar@/^2pc/[ll]^{\txt{git checkout}}&
          *++[F-,]{\txt{Remote Repo}} \ar@/^2pc/[l]^{\txt{git fetch}} \ar `d[dd] `[ddlll]^{\txt{git pull}} [lll] \\
          & & & \\
          & & &
  }}\end{figure}}
\end{frame}

\section{Merging and Rebasing}

\subsection{Merging}

\begin{frame}
  \frametitle{Fast Forward Merging}
  \begin{itemize}
    \item \texttt{git checkout master}
    \item \texttt{git merge SAVE-234}
  \end{itemize}
  \only<1>{
    \dia {
      *++[o][F-]{a} \ar[r] &
      *++[o][F-]{b} \ar[r] \save +<0cm,0.6cm>*\txt{Master} \restore &
      *++[o][F-]{c} \ar[r] &
      *++[o][F-]{d} \ar[r] &
      *++[o][F-]{e} \save +<0cm,0.6cm>*\txt{SAVE-234} \restore
  }}
  \only<2>{
    \dia {
      *++[o][F-]{a} \ar[r] &
      *++[o][F-]{b} \ar[r] \ar@/_2pc/[rrr]_{Master} &
      *++[o][F-]{c} \ar[r] &
      *++[o][F-]{d} \ar[r] &
      *++[o][F-]{e} \save +<0cm,0.6cm>*\txt{SAVE-234} \restore
  }}
  \only<3>{
    \dia {
    *++[o][F-]{a} \ar[r] &
    *++[o][F-]{b} \ar[r] &
    *++[o][F-]{c} \ar[r] &
    *++[o][F-]{d} \ar[r] &
    *++[o][F-]{e} \save +<0cm,-0.6cm>*\txt{Master} \restore \save +<0cm,0.6cm>*\txt{SAVE-234} \restore
  }}
\end{frame}
\begin{frame}
  \frametitle{Merging}
  \begin{itemize}
    \item \texttt{git checkout master}
    \item \texttt{git merge SAVE-234}
  \end{itemize}
  \only<1> { \dia {
    *++[o][F-]{a} \ar[r] &
    *++[o][F-]{b} \ar[r] \ar[rd] &
    *++[o][F-]{c} \ar[r] &
    *++[o][F-]{d} \ar[r] &
    *++[o][F-]{e} \save +<0cm,0.6cm>*\txt{Master} \restore \\ & &
    *++[o][F-]{g} \ar[r] &
    *++[o][F-]{h} \ar[r] &
    *++[o][F-]{i} \save +<0cm,0.6cm>*\txt{SAVE-234} \restore
  }}
  \only<2> { \dia {
    *++[o][F-]{a} \ar[r] &
    *++[o][F-]{b} \ar[r] \ar[rd] &
    *++[o][F-]{c} \ar[r] &
    *++[o][F-]{d} \ar[r] &
    *++[o][F-]{e} \ar[r] &
    *++[o][F=]{j}  \save +<0cm,0.6cm>*\txt{Master} \restore \\ & &
    *++[o][F-]{g} \ar[r] &
    *++[o][F-]{h} \ar[r] &
    *++[o][F-]{i} \ar[ru] \save +<0cm,-0.6cm>*\txt{SAVE-234} \restore
  }}
\end{frame}

\begin{frame}
  \frametitle{No Fast Forward Merging}
  \begin{itemize}
    \item \texttt{git checkout master}
    \item \texttt{git merge --no-ff SAVE-234}
  \end{itemize}
  \only<1> {
    \dia {
      *++[o][F-]{a} \ar[r] &
      *++[o][F-]{b} \ar[r] \save +<0cm,0.6cm>*\txt{Master} \restore &
      *++[o][F-]{c} \ar[r] &
      *++[o][F-]{d} \ar[r] &
      *++[o][F-]{e} \save +<0cm,0.6cm>*\txt{SAVE-234} \restore
  }}
  \only<2> {
    \dia {
      *++[o][F-]{a} \ar[r] &
      *++[o][F-]{b} \ar[rd] \ar[rrrr] & & & &
      *++[o][F=]{f} \save +<0cm,0.6cm>*\txt{Master} \restore \\ & &
      *++[o][F-]{c} \ar[r] &
      *++[o][F-]{d} \ar[r] &
      *++[o][F-]{e} \ar[ru] \save +<0cm,-0.6cm>*\txt{SAVE-234} \restore
  }}
\end{frame}

\subsection{Rebase}

\begin{frame}
  \frametitle{Rebase}
  \begin{itemize}
    \item \texttt{git checkout SAVE-}234
    \item \texttt{git rebase master}
  \end{itemize}
  \only<1> { \dia {
    *++[o][F-]{a} \ar[r] &
    *++[o][F-]{b} \ar[r] \ar[rd] &
    *++[o][F-]{c} \ar[r] &
    *++[o][F-]{d} \ar[r] &
    *++[o][F-]{e} \save +<0cm,0.6cm>*\txt{Master} \restore \\ & &
    *++[o][F-]{g} \ar[r] &
    *++[o][F-]{h} \ar[r] &
    *++[o][F-]{i} \save +<0cm,0.6cm>*\txt{SAVE-234} \restore
  }}
  \only<2> { \dia {
    *++[o][F-]{a} \ar[r] &
    *++[o][F-]{b} \ar[r] &
    *++[o][F-]{c} \ar[r] &
    *++[o][F-]{d} \ar[r] &
    *++[o][F-]{e} \ar[lld] \save +<0cm,0.6cm>*\txt{Master} \restore \\ & &
    *++[o][F=]{g'} \ar[r] &
    *++[o][F=]{h'} \ar[r] &
    *+++[o][F=]{i'} \save +<0cm,-0.8cm>*\txt{SAVE-234} \restore
  }}
\end{frame}

\section{Extras}

\subsection{Remotes}

\begin{frame}
  \frametitle{Remote Commands}
  \begin{itemize}
  \item \texttt{git remote add origin git@scm.savewave.com:savingstar/savingstar-web.git}
    \\ This creates a new remote named origin linked up to gitorious
    \pause
  \item \texttt{git fetch}
    \\ Pulls in objects from remotes to local object store
    \pause
  \item \texttt{git branch -a}
    \\ Shows all branches including remote copies
  \end{itemize}
\end{frame}

\begin{frame}
  \frametitle{Pulling}
  \begin{itemize}
  \item Special case
  \item Executes:
    \begin{itemize}
    \item \texttt{git fetch}
    \item \texttt{git merge}
    \end{itemize}
  \item Only performs the merge for the current branch
  \end{itemize}
\end{frame}

\begin{frame}
  \frametitle{Pushing}
  \begin{itemize}
    \item \texttt{git push origin source:destination} \\ This will push objects
      and update the destination branch with objects from the source
      \pause
    \item \texttt{git push origin :destination} \\ This will delete the remote
      branch from the server
      \pause
    \item \texttt{git push -f} \\ Push normally doesn't allow anything but fast
      forward, while this completely overrides the destination using the source
      objects
  \end{itemize}
\end{frame}

\subsection{Misc}

\begin{frame}
  \frametitle{Git Config File}
  \begin{itemize}
    \item Push defaults
    \item Commiter name and email
    \item Color settings
    \item Aliases
  \end{itemize}
\end{frame}

\begin{frame}
  \frametitle{Some Tips}
  \begin{itemize}
    \item Always work on branches
      \pause
    \item When in doubt use GitX or gitk to see what you are doing
      \pause
    \item When in doubt save a new branch so you can always get back
      \pause
    \item Add the current git branch into your prompt
      \pause
    \item Explore with rebase and cleaning up code before final push
  \end{itemize}
\end{frame}

\end{document}
