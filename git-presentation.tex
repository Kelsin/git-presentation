\usepackage{color}
\usepackage{float}
\usepackage{hyperref}

\input xy
\xyoption{all}

\usepackage{beamerthemesplit}
\usetheme{Darmstadt}
\usecolortheme{crane}

\setbeamercovered{dynamic}

\title{Git}
\subtitle{https://github.com/kelsin/git-presentation}
\author{Christopher Giroir}
\date{June 1st, 2023}

% Commands
\newcommand{\dia}[1]{\begin{figure}[H]\centerline{\xymatrix{#1}}\end{figure}}
\newcommand{\code}[2]{\ttfamily\begin{block}{\textnormal{#1}}#2\end{block}\normalfont}

\begin{document}

\maketitle

\begin{frame}
  \titlepage
\end{frame}

\begin{frame}
  \tableofcontents[hideallsubsections]
\end{frame}

\begin{frame}
  \frametitle{?s}
  Please interupt with questions at any time.
\end{frame}

\section{History}

\subsection{Creation}

\begin{frame}
  \frametitle{Creation}
  \begin{itemize}
  \item Created by Linus Torvalds
  \pause
  \item Handed off after 3 months
  \pause
  \item BitKeeper removed Linux's free license
  \pause
  \item Built for distributed workflows and speed
  \end{itemize}
\end{frame}

\subsection{Competitors}

\begin{frame}
  \frametitle{Competitors}
  \begin{itemize}
  \item \href{https://www.nongnu.org/cvs/}{CVS} / \href{https://subversion.apache.org/}{SVN}
  \pause
  \item \href{http://darcs.net/}{Darcs} / \href{https://en.wikipedia.org/wiki/GNU_Bazaar}{Bazaar}
  \pause
  \item \href{https://www.mercurial-scm.org/}{Mercurial}
  \end{itemize}
\end{frame}

\subsection{Github}

\begin{frame}
  \frametitle{Github}
  \begin{itemize}
  \item \href{https://github.com/}{Github} is a web front end on top of \href{https://git-scm.com/}{Git}
  \pause
  \item Very popular
  \pause
  \item This talk: core git
  \end{itemize}
\end{frame}

\section{Trees and Blobs}

\subsection{Objects}

\begin{frame}
  \frametitle{Object Store}
  \begin{itemize}
  \item Black Box!
  \pause
  \item Speedy and Correct
  \pause
  \item Comparisons
    \begin{itemize}
    \item More information than CVS
    \item More sane than SVN
    \item More efficient than Mercurial
    \end{itemize}
  \end{itemize}
\end{frame}

In my experience people seem to assume things about the way git works. Often
these assumptions are wrong. ``Doing it that way would be slow'' is often
heard. While learning I think it's good to ignore these issues and just believe
that the object store is amazing and allows git to perform very nicely.

That being said, everything in the .git folder EXCEPT the object store is plain
text files. Often all the tools do is manipulate them. Feel free to poke around
and explore in the .git directory.

\begin{frame}
  \frametitle{Blobs}
  \dia{ *+++=[o][F-]{\txt{\textbf{Blob}}} }
  \begin{itemize}
  \item Store text or binary data
  \item Labeled by SHA1
  \end{itemize}
\end{frame}

\begin{frame}
  \frametitle{Trees}
  \dia{ &
    *+++=[o][F-]{\txt{Blob}} \\
    *+++=[o][F-]{\txt{\textbf{Tree}}} \ar[r] \ar[ru] &
    *+++=[o][F-]{\txt{Blob}}
  }
  \begin{itemize}
  \item Trees point to blobs or other trees
  \item Represent the entire state of the directory
  \item Labeled by SHA1
  \end{itemize}
\end{frame}

\begin{frame}
  \frametitle{Commits}
  \dia {
    *+++=[o][F-]{\txt{Commit}} & &
    *+++=[o][F-]{\txt{Blob}} \\
    *+++=[o][F-]{\txt{\textbf{Commit}}} \ar[u] \ar[r] &
    *+++=[o][F-]{\txt{Tree}} \ar[ru] \ar[r] &
    *+++=[o][F-]{\txt{Blob}}
  }
  \begin{itemize}
  \item 0 to N parents
  \item Points to 1 tree
  \item Metadata: author, date, \textit{etc}
  \item Labeled by SHA1
  \end{itemize}
\end{frame}

\begin{frame}
  \frametitle{Important Distinction!}
  \begin{itemize}
  \item Commits link to \textcolor{blue}{\textbf{TREES}} not \textcolor{red}{\textbf{DIFFS}}.
  \item Diffs are not stored in git, they are computed.
  \end{itemize}
\end{frame}

I want to emphasize this point. Don't think in terms of diffs. In a linear
system (like SVN or CVS) this was an easy abstraction to use. In some systems
(like darcs) this is how the systems work. In git it's not. You have to think in
terms of commits (linking to whole trees) in order for anything else to make
sense.

\subsection{Branches and Tags}

\begin{frame}
  \frametitle{Pointers}
  Most other things in git are pointers to commits.
  \dia {
    *++[o][F-]{a} \ar[r] \save +<0cm,0.6cm>*\txt{\textit{v3.0.1}} \restore &
    *++[o][F-]{b} \ar[r] \ar[rd] \save +<0cm,0.6cm>*\txt{main} \restore &
    *++[o][F-]{c} \ar[r] &
    *++[o][F-]{d} \ar[r] &
    *++[o][F-]{e} \save +<0cm,0.6cm>*\txt{foo} \restore \\ & &
    *++[o][F-]{g} \ar[r] \save +<0cm,-0.6cm>*\txt{origin/bar} \restore &
    *++[o][F-]{h} \ar[r] &
    *++[o][F-]{i} \save +<0cm,-0.6cm>*\txt{bar} \restore
  }
  \begin{itemize}
  \item Branches are pointers to commits
  \item Tags also include meta information
  \end{itemize}
\end{frame}

\begin{frame}
  \frametitle{Rewriting History}
  \begin{itemize}
  \item Rewriting a commit changes it's SHA
    \pause
  \item Any commit after it changes as well
    \pause
  \item Don't be afraid
  \end{itemize}
\end{frame}

\section{Locations}

\subsection{Advantages}

\begin{frame}
  \frametitle{Distributed Source Control}
  \begin{itemize}
  \item Remote and Local differences
    \pause
  \item Can function offline
    \pause
  \item Can ``push'' and ``pull'' from each other
    \pause
  \item We are all backups
  \end{itemize}
\end{frame}

\subsection{Architecture}

\begin{frame}
  \frametitle{Location Descriptions}
  Git has 4 locations where commits are stored
  \pause
  \begin{itemize}
  \item \textbf{Remote Repo} Any remote object store
    \pause
  \item \textbf{Local Repo} Your local object store
    \pause
  \item \textbf{Index} Place to ``stage'' commits
    \pause
  \item \textbf{Working Directory} Your local files
  \end{itemize}
\end{frame}

These are logical locations, not file locations.

\subsection{Command Examples}

\begin{frame}
  \frametitle{Locations and Commands}

  \only<beamer|beamer:1>{
    \dia {
          *++[F-,]{\txt{Working}} &
          *++[F-,]{\txt{Index}} &
          *++[F-,]{\txt{Local Repo}} &
          *++[F-,]{\txt{Remote Repo}}
  }}
  \only<beamer|beamer:2>{
    \dia {
          *++[F-,]{\txt{Working}} \ar@/^2pc/[r]^{\txt{git add}}&
          *++[F-,]{\txt{Index}} \ar@/^2pc/[r]^{\txt{git commit}}&
          *++[F-,]{\txt{Local Repo}} &
          *++[F-,]{\txt{Remote Repo}}
  }}
  \only<beamer|beamer:3>{
    \dia {
          *++[F-,]{\txt{Working}} \ar@/^2pc/[r]^{\txt{git add}}&
          *++[F-,]{\txt{Index}} \ar@/^2pc/[r]^{\txt{git commit}}&
          *++[F-,]{\txt{Local Repo}} \ar@/^2pc/[ll]^{\txt{git checkout}}&
          *++[F-,]{\txt{Remote Repo}}
  }}
  \only<4>{
    \dia {
          *++[F-,]{\txt{Working}} \ar@/^2pc/[r]^{\txt{git add}}&
          *++[F-,]{\txt{Index}} \ar@/^2pc/[r]^{\txt{git commit}}&
          *++[F-,]{\txt{Local Repo}} \ar@/^2pc/[r]^{\txt{git push}} \ar@/^2pc/[ll]^{\txt{git checkout}}&
          *++[F-,]{\txt{Remote Repo}} \ar@/^2pc/[l]^{\txt{git fetch}} \ar `d[dd] `[ddlll]^{\txt{git pull}} [lll] \\
          & & & \\
          & & &
  }}
\end{frame}

\begin{frame}
  \frametitle{Pulling}
  \code{Pull in remote changes}{git pull origin main}
  \begin{itemize}
  \item \textcolor{red}{Special case}
  \item Executes:
    \begin{itemize}
    \item \texttt{git fetch}
    \item \texttt{git merge}
    \end{itemize}
  \item Merges in to your current branch
  \end{itemize}
\end{frame}

\section{Merging and Rebasing}

\subsection{Merging}

\begin{frame}
  \frametitle{Fast Forward Merging}
  \begin{itemize}
    \item \texttt{git checkout main}
    \item \texttt{git merge foo}
  \end{itemize}
  \only<article|beamer|beamer:1>{
    \dia {
      *++[o][F-]{a} \ar[r] &
      *++[o][F-]{b} \ar[r] \save +<0cm,0.6cm>*\txt{main} \restore &
      *++[o][F-]{c} \ar[r] &
      *++[o][F-]{d} \ar[r] &
      *++[o][F-]{e} \save +<0cm,-0.6cm>*\txt{foo} \restore
  }}
  \only<2>{
    \dia {
    *++[o][F-]{a} \ar[r] &
    *++[o][F-]{b} \ar[r] &
    *++[o][F-]{c} \ar[r] &
    *++[o][F-]{d} \ar[r] &
    *++[o][F-]{e} \save +<0cm,0.6cm>*\txt{main} \restore \save +<0cm,-0.6cm>*\txt{foo} \restore
  }}
\end{frame}

\begin{frame}
  \frametitle{Merging}
  \begin{itemize}
    \item \texttt{git checkout main}
    \item \texttt{git merge foo}
  \end{itemize}
  \only<article|beamer|beamer:1> { \dia {
    *++[o][F-]{a} \ar[r] &
    *++[o][F-]{b} \ar[r] \ar[rd] &
    *++[o][F-]{c} \ar[r] &
    *++[o][F-]{d} \ar[r] &
    *++[o][F-]{e} \save +<0cm,0.6cm>*\txt{main} \restore \\ & &
    *++[o][F-]{g} \ar[r] &
    *++[o][F-]{h} \ar[r] &
    *++[o][F-]{i} \save +<0cm,-0.6cm>*\txt{foo} \restore
  }}
  \only<2> { \dia {
    *++[o][F-]{a} \ar[r] &
    *++[o][F-]{b} \ar[r] \ar[rd] &
    *++[o][F-]{c} \ar[r] &
    *++[o][F-]{d} \ar[r] &
    *++[o][F-]{e} \ar[r] &
    *++[o][F=]{j}  \save +<0cm,0.6cm>*\txt{main} \restore \\ & &
    *++[o][F-]{g} \ar[r] &
    *++[o][F-]{h} \ar[r] &
    *++[o][F-]{i} \ar[ru] \save +<0cm,-0.6cm>*\txt{foo} \restore
  }}
\end{frame}

\begin{frame}
  \frametitle{No Fast Forward Merging}
  \begin{itemize}
    \item \texttt{git checkout main}
    \item \texttt{git merge --no-ff foo}
  \end{itemize}
  \only<article|beamer|beamer:1> {
    \dia {
      *++[o][F-]{a} \ar[r] &
      *++[o][F-]{b} \ar[r] \save +<0cm,0.6cm>*\txt{main} \restore &
      *++[o][F-]{c} \ar[r] &
      *++[o][F-]{d} \ar[r] &
      *++[o][F-]{e} \save +<0cm,-0.6cm>*\txt{foo} \restore
  }}
  \only<2> {
    \dia {
      *++[o][F-]{a} \ar[r] &
      *++[o][F-]{b} \ar[rd] \ar[rrrr] & & & &
      *++[o][F=]{f} \save +<0cm,0.6cm>*\txt{main} \restore \\ & &
      *++[o][F-]{c} \ar[r] &
      *++[o][F-]{d} \ar[r] &
      *++[o][F-]{e} \ar[ru] \save +<0cm,-0.6cm>*\txt{foo} \restore
  }}
\end{frame}

\subsection{Rebase}

\begin{frame}
  \frametitle{Rebase}
  \begin{itemize}
    \item \texttt{git checkout foo}
    \item \texttt{git rebase main}
  \end{itemize}
  \only<article|beamer|beamer:1> { \dia {
    *++[o][F-]{a} \ar[r] &
    *++[o][F-]{b} \ar[r] \ar[rd] &
    *++[o][F-]{c} \ar[r] &
    *++[o][F-]{d} \ar[r] &
    *++[o][F-]{e} \save +<0cm,0.6cm>*\txt{main} \restore \\ & &
    *++[o][F-]{g} \ar[r] &
    *++[o][F-]{h} \ar[r] &
    *++[o][F-]{i} \save +<0cm,-0.6cm>*\txt{foo} \restore
  }}
  \only<2> { \dia {
    *++[o][F-]{a} \ar[r] &
    *++[o][F-]{b} \ar[r] &
    *++[o][F-]{c} \ar[r] &
    *++[o][F-]{d} \ar[r] &
    *++[o][F-]{e} \ar[lld] \save +<0cm,0.6cm>*\txt{main} \restore \\ & &
    *++[o][F=]{g'} \ar[r] &
    *++[o][F=]{h'} \ar[r] &
    *+++[o][F=]{i'} \save +<0cm,-0.8cm>*\txt{foo} \restore
  }}
\end{frame}

\section{Extras}

\subsection{Git Config File}

\begin{frame}
  \frametitle{Git Config File}
  My git config is available \href{https://github.com/kelsin/dotfiles/blob/main/main/.gitconfig}{on github}
  \begin{itemize}
    \item Push defaults
    \item Commiter name and email
    \item Color settings
    \item Aliases
  \end{itemize}
\end{frame}

\subsection{Tips}

\begin{frame}
  \frametitle{Tips}
  \begin{itemize}
    \item Always work on branches
      \pause
    \item Use GitX or gitk to see what you are doing
      \pause
    \item Save a new branch as a backup
      \pause
    \item Add git branch to your shell prompt
      \pause
    \item Explore with rebasing and cleaning!
  \end{itemize}
\end{frame}

\begin{frame}
  \frametitle{Questions?}
  Questions?
\end{frame}

\begin{frame}
  \frametitle{The End}
  The End
\end{frame}

\end{document}
