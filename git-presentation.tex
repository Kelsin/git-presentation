\usepackage{color}
\usepackage{float}

\input xy
\xyoption{all}

\usepackage{beamerthemesplit}
\usetheme{Darmstadt}
\usecolortheme{crane}

\setbeamercovered{dynamic}

\title{Git for mere mortals}
\subtitle{https://github.com/Kelsin/git-presentation}
\author{Christopher Giroir}
\date{July 26th, 2011}

% Commands
\newcommand{\dia}[1]{\begin{figure}[H]\centerline{\xymatrix{#1}}\end{figure}}
\newcommand{\code}[2]{\ttfamily\begin{block}{\textnormal{#1}}#2\end{block}\normalfont}

\begin{document}

\maketitle

\begin{frame}
  \titlepage
\end{frame}

\begin{frame}
  \tableofcontents
\end{frame}

\section{Internals}

\subsection{Objects}

\begin{frame}
  \frametitle{Object Store}
  \begin{itemize}
  \item The git object store is a big black box and we like it that way.
  \pause
  \item It was written for speed and correctness
  \pause
  \item Comparisons
    \begin{itemize}
    \item More information than CVS
    \item More sane than SVN
    \item More efficient than Mercurial
    \end{itemize}
  \end{itemize}
\end{frame}

In my experience people seem to assume things about the way git works. Often
these assumptions are wrong. ``Doing it that way would be slow'' is often
heard. While learning I think it's good to ignore these issues and just believe
that the object store is amazing and allows git to perform very nicely.

That being said, everything in the .git folder EXCEPT the object store is plain
text files. Often all the tools do is manipulate them. Feel free to poke around
and explore in the .git directory.

\begin{frame}
  \frametitle{Blobs}
  \dia{ *+++=[o][F-]{\txt{\textbf{Blob}}} }
  \begin{itemize}
  \item Blobs store file data (text or binary).
  \item Labeled by SHA1. \textit{If the content changes, so does the SHA.}
  \end{itemize}
\end{frame}

\begin{frame}
  \frametitle{Trees}
  \dia{ &
    *+++=[o][F-]{\txt{Blob}} \\
    *+++=[o][F-]{\txt{\textbf{Tree}}} \ar[r] \ar[ru] &
    *+++=[o][F-]{\txt{Blob}}
  }
  \begin{itemize}
  \item Trees point to blobs
  \item Represent the entire state of the working directory
  \item Also labeled by SHA's.
  \end{itemize}
\end{frame}

\begin{frame}
  \frametitle{Commits}
  \dia {
    *+++=[o][F-]{\txt{Commit}} & &
    *+++=[o][F-]{\txt{Blob}} \\
    *+++=[o][F-]{\txt{\textbf{Commit}}} \ar[u] \ar[r] &
    *+++=[o][F-]{\txt{Tree}} \ar[ru] \ar[r] &
    *+++=[o][F-]{\txt{Blob}}
  }
  \begin{itemize}
  \item Points to a tree object and parent commit(s)
  \item Stores author and date information
  \item Labeled by SHA's.
  \end{itemize}
\end{frame}

\begin{frame}
  \frametitle{Important Distinction!}
  \begin{itemize}
  \item Commits link to entires \textcolor{blue}{\textbf{TREES}} not
  \textcolor{red}{\textbf{DIFFS}}.
  \item Diffs are not stored in git, they are computed.
  \end{itemize}
\end{frame}

I want to emphasize this point. Don't think in terms of diffs. In a linear
system (like SVN or CVS) this was an easy abstraction to use. In some systems
(like darcs) this is how the systems work. In git it's not. You have to think in
terms of commits (linking to whole trees) in order for anything else to make
sense.

\subsection{Branches and Tags}

\begin{frame}
  \frametitle{Pointers}
  Most other things in git are pointers to commits.
  \dia {
    *++[o][F-]{a} \ar[r] \save +<0cm,0.6cm>*\txt{\textit{v3.0.1}} \restore &
    *++[o][F-]{b} \ar[r] \ar[rd] \save +<0cm,0.6cm>*\txt{Master} \restore &
    *++[o][F-]{c} \ar[r] &
    *++[o][F-]{d} \ar[r] &
    *++[o][F-]{e} \save +<0cm,0.6cm>*\txt{SAVE-465} \restore \\ & &
    *++[o][F-]{g} \ar[r] \save +<0cm,-0.6cm>*\txt{origin/SAVE-576} \restore &
    *++[o][F-]{h} \ar[r] &
    *++[o][F-]{i} \save +<0cm,-0.6cm>*\txt{SAVE-576} \restore
  }
  \begin{itemize}
  \item Branches are just labels to commit SHA's
  \item Tags are just labels with meta information
  \end{itemize}
\end{frame}

\begin{frame}
  \frametitle{Rewriting History}
  \begin{itemize}
  \item Rewriting a commit changes it's SHA
    \pause
  \item Any commit after it changes as well
    \pause
  \item Most branch operations are quick and painless
    \pause
  \item Can shoot yourself in the foot and heal it back up again
  \end{itemize}
\end{frame}

\section{Usage}

I don't want to spend a lot of time on examples since you can find that
information in cheat sheets.

\subsection{Creating and Commiting}

\begin{frame}
  \frametitle{Starting Out}
  \code{Setting User Info}{
git config --global user.name "Christopher Giroir" \newline
git config --global user.email "kelsin@valefor.com"}
    \pause
  \code{Completely New Repo}{
cd dir-with-code \newline
git init}
    \pause
  \code{Cloning any accessible repo}{
git clone git@github.com:Kelsin/configs.git \newline
cd configs}
\end{frame}

\begin{frame}
  \frametitle{Commits}
  \code{Initial Commit}{
git add . \newline
git commit -m "Initial Commit"}
    \pause
  \code{Editing some files}{
git add edited-file.rb \newline
git commit -m "Improved everything"}
    \pause
  \code{Adding to previous commit}{
git add edited-file.rb \newline
git commit --amend}
\end{frame}

Ammending does rewrite history so be careful if you are dealing with others and
pushed code.

\subsection{Inspection}

\begin{frame}
  \frametitle{Status and Diffs}
  \code{Normal Status}{git status}
    \pause
  \code{What's current changed?}{git diff}
    \pause
  \code{Changes from only one file}{git diff example-file.rb}
    \pause
  \code{Changes between branches}{git diff master}
\end{frame}

Most git commands page automatically in an interactive terminal. Git diff
accepst almost all flags that normal diff accepts.

\begin{frame}
  \frametitle{Show and Log}
  \code{Show current commit}{git show}
    \pause
  \code{Show some other commit}{git show 234ab32}
    \pause
  \code{Show log of recent commits}{git log}
    \pause
  \code{Show pretty log}{git log --graph --decorate --oneline}
\end{frame}

Git has many options for show commits both in log form and in individual detail.

\begin{frame}
  \frametitle{Refs and Objects}
  \code{By commit id}{git show 234ab32}
    \pause
  \code{By branch}{git show master}
    \pause
  \code{By tag}{git show v3.0.2}
    \pause
  \code{Relative}{git show HEAD\^{}}
\end{frame}

You can reference git commits in many ways including dates, branch labels, tag
labels, etc. There are even some commands for finding commits relative to other
commits.

\section{Distribution}

\subsection{Advantages}

\begin{frame}
  \frametitle{Distributed Source Control}
  \begin{itemize}
  \item Only difference from remote and local (normally): working directories
    \pause
  \item Can function offline
    \pause
  \item Can ``push'' and ``pull'' from each other, as well as servers
    \pause
  \item Each of our computers serves as a backup of the server
  \end{itemize}
\end{frame}

\subsection{Architecture}

\begin{frame}
  \frametitle{Location Descriptions}
  Git has 4 locations where commits are stored
  \pause
  \begin{itemize}
  \item \textbf{Remote Repo} Any remote object store
    \pause
  \item \textbf{Local Repo} You local object store
    \pause
  \item \textbf{Index} A single spot to ``stage'' commits
    \pause
  \item \textbf{Working Directory} These are the files you are editing
  \end{itemize}
\end{frame}

\subsection{Command Examples}

\begin{frame}
  \frametitle{Locations and Commands}

  \only<1>{
    \dia {
          *++[F-,]{\txt{Working}} &
          *++[F-,]{\txt{Index}} &
          *++[F-,]{\txt{Local Repo}} &
          *++[F-,]{\txt{Remote Repo}}
  }}
  \only<2>{
    \dia {
          *++[F-,]{\txt{Working}} \ar@/^2pc/[r]^{\txt{git add}}&
          *++[F-,]{\txt{Index}} \ar@/^2pc/[r]^{\txt{git commit}}&
          *++[F-,]{\txt{Local Repo}} &
          *++[F-,]{\txt{Remote Repo}}
  }}
  \only<3>{
    \dia {
          *++[F-,]{\txt{Working}} \ar@/^2pc/[r]^{\txt{git add}}&
          *++[F-,]{\txt{Index}} \ar@/^2pc/[r]^{\txt{git commit}}&
          *++[F-,]{\txt{Local Repo}} \ar@/^2pc/[ll]^{\txt{git checkout}}&
          *++[F-,]{\txt{Remote Repo}}
  }}
  \only<4>{
    \dia {
          *++[F-,]{\txt{Working}} \ar@/^2pc/[r]^{\txt{git add}}&
          *++[F-,]{\txt{Index}} \ar@/^2pc/[r]^{\txt{git commit}}&
          *++[F-,]{\txt{Local Repo}} \ar@/^2pc/[r]^{\txt{git push}} \ar@/^2pc/[ll]^{\txt{git checkout}}&
          *++[F-,]{\txt{Remote Repo}} \ar@/^2pc/[l]^{\txt{git fetch}} \ar `d[dd] `[ddlll]^{\txt{git pull}} [lll] \\
          & & & \\
          & & &
  }}
\end{frame}

\section{Merging and Rebasing}

\subsection{Merging}

\begin{frame}
  \frametitle{Fast Forward Merging}
  \begin{itemize}
    \item \texttt{git checkout master}
    \item \texttt{git merge SAVE-234}
  \end{itemize}
  \only<1>{
    \dia {
      *++[o][F-]{a} \ar[r] &
      *++[o][F-]{b} \ar[r] \save +<0cm,0.6cm>*\txt{Master} \restore &
      *++[o][F-]{c} \ar[r] &
      *++[o][F-]{d} \ar[r] &
      *++[o][F-]{e} \save +<0cm,-0.6cm>*\txt{SAVE-234} \restore
  }}
  \only<2>{
    \dia {
    *++[o][F-]{a} \ar[r] &
    *++[o][F-]{b} \ar[r] &
    *++[o][F-]{c} \ar[r] &
    *++[o][F-]{d} \ar[r] &
    *++[o][F-]{e} \save +<0cm,0.6cm>*\txt{Master} \restore \save +<0cm,-0.6cm>*\txt{SAVE-234} \restore
  }}
\end{frame}

\begin{frame}
  \frametitle{Merging}
  \begin{itemize}
    \item \texttt{git checkout master}
    \item \texttt{git merge SAVE-234}
  \end{itemize}
  \only<1> { \dia {
    *++[o][F-]{a} \ar[r] &
    *++[o][F-]{b} \ar[r] \ar[rd] &
    *++[o][F-]{c} \ar[r] &
    *++[o][F-]{d} \ar[r] &
    *++[o][F-]{e} \save +<0cm,0.6cm>*\txt{Master} \restore \\ & &
    *++[o][F-]{g} \ar[r] &
    *++[o][F-]{h} \ar[r] &
    *++[o][F-]{i} \save +<0cm,-0.6cm>*\txt{SAVE-234} \restore
  }}
  \only<2> { \dia {
    *++[o][F-]{a} \ar[r] &
    *++[o][F-]{b} \ar[r] \ar[rd] &
    *++[o][F-]{c} \ar[r] &
    *++[o][F-]{d} \ar[r] &
    *++[o][F-]{e} \ar[r] &
    *++[o][F=]{j}  \save +<0cm,0.6cm>*\txt{Master} \restore \\ & &
    *++[o][F-]{g} \ar[r] &
    *++[o][F-]{h} \ar[r] &
    *++[o][F-]{i} \ar[ru] \save +<0cm,-0.6cm>*\txt{SAVE-234} \restore
  }}
\end{frame}

\begin{frame}
  \frametitle{No Fast Forward Merging}
  \begin{itemize}
    \item \texttt{git checkout master}
    \item \texttt{git merge --no-ff SAVE-234}
  \end{itemize}
  \only<1> {
    \dia {
      *++[o][F-]{a} \ar[r] &
      *++[o][F-]{b} \ar[r] \save +<0cm,0.6cm>*\txt{Master} \restore &
      *++[o][F-]{c} \ar[r] &
      *++[o][F-]{d} \ar[r] &
      *++[o][F-]{e} \save +<0cm,-0.6cm>*\txt{SAVE-234} \restore
  }}
  \only<2> {
    \dia {
      *++[o][F-]{a} \ar[r] &
      *++[o][F-]{b} \ar[rd] \ar[rrrr] & & & &
      *++[o][F=]{f} \save +<0cm,0.6cm>*\txt{Master} \restore \\ & &
      *++[o][F-]{c} \ar[r] &
      *++[o][F-]{d} \ar[r] &
      *++[o][F-]{e} \ar[ru] \save +<0cm,-0.6cm>*\txt{SAVE-234} \restore
  }}
\end{frame}

\subsection{Rebase}

\begin{frame}
  \frametitle{Rebase}
  \begin{itemize}
    \item \texttt{git checkout SAVE-}234
    \item \texttt{git rebase master}
  \end{itemize}
  \only<1> { \dia {
    *++[o][F-]{a} \ar[r] &
    *++[o][F-]{b} \ar[r] \ar[rd] &
    *++[o][F-]{c} \ar[r] &
    *++[o][F-]{d} \ar[r] &
    *++[o][F-]{e} \save +<0cm,0.6cm>*\txt{Master} \restore \\ & &
    *++[o][F-]{g} \ar[r] &
    *++[o][F-]{h} \ar[r] &
    *++[o][F-]{i} \save +<0cm,-0.6cm>*\txt{SAVE-234} \restore
  }}
  \only<2> { \dia {
    *++[o][F-]{a} \ar[r] &
    *++[o][F-]{b} \ar[r] &
    *++[o][F-]{c} \ar[r] &
    *++[o][F-]{d} \ar[r] &
    *++[o][F-]{e} \ar[lld] \save +<0cm,0.6cm>*\txt{Master} \restore \\ & &
    *++[o][F=]{g'} \ar[r] &
    *++[o][F=]{h'} \ar[r] &
    *+++[o][F=]{i'} \save +<0cm,-0.8cm>*\txt{SAVE-234} \restore
  }}
\end{frame}

\section{Extras}

\subsection{Remotes}

\begin{frame}
  \frametitle{Remote Commands}
  \code{Adding remote}{git remote add origin git@github.com:Kelsin/configs.git}
    \pause
  \code{Fetch all remote objects}{git fetch}
    \pause
  \code{Show all branches}{git branch -a}
\end{frame}

\begin{frame}
  \frametitle{Pushing}
  \code{Push objects from source to destination}{git push origin source:destination}
    \pause
  \code{Removing remote branch}{git push origin :destination}
    \pause
  \code{Setting upstream}{git push -u origin SAVE-453}
    \pause
  \code{Force pushes}{git push -f origin SAVE-453}
\end{frame}

Force pushing and removing remote branches is a dangerous operation. Make
backups of branches any time you are worried.

\begin{frame}
  \frametitle{Pulling}
  \code{Pull in remote changes}{git pull origin master}
  \begin{itemize}
  \item \textcolor{red}{Special case}
  \item Executes:
    \begin{itemize}
    \item \texttt{git fetch}
    \item \texttt{git merge}
    \end{itemize}
  \item Merges in to your current branch
  \end{itemize}
\end{frame}

\subsection{Misc}

\begin{frame}
  \frametitle{Git Config File}
  My git config is available at https://github.com/Kelsin/configs/blob/master/.gitconfig
  \begin{itemize}
    \item Push defaults
    \item Commiter name and email
    \item Color settings
    \item Aliases
  \end{itemize}
\end{frame}

\begin{frame}
  \frametitle{Some Tips}
  \begin{itemize}
    \item Always work on branches
      \pause
    \item When in doubt use GitX or gitk to see what you are doing
      \pause
    \item When in doubt save a new branch so you can always get back
      \pause
    \item Add the current git branch into your prompt
      \pause
    \item Explore with rebase and cleaning up code before final push
  \end{itemize}
\end{frame}

\end{document}
